% 请使用 xelatex 编译

\documentclass[14pt,a4paper]{moderncv}

%主题设置
\moderncvstyle{classic}                     % 选项参数是 ‘casual’, ‘classic’, ‘oldstyle’ 和 ’banking’
\moderncvcolor{blue} % optional argument are 'blue' (default), 'orange', 'red', 'green', 'grey' and 'roman' (for roman fonts, instead of sans serif fonts)


\usepackage{fontspec,xunicode,xltxtra}
% 设置英文字体
\defaultfontfeatures{Scale=MatchLowercase} % 设置字体大小,Scale=0.8等
\setmainfont{Times New Roman} %主字体
\setsansfont{Times New Roman} %无衬线字体
\setmonofont{Times New Roman} %等宽字体
% 设置中文字体
\usepackage[BoldFont,SlantFont,CJKchecksingle,CJKnumber,CJKtextspaces]{xeCJK}
\setCJKmainfont{Adobe Heiti Std} %主要中文字体
\setCJKsansfont{Adobe Heiti Std} %无衬线中文字体
\setCJKmonofont{Adobe Heiti Std} %等宽中文字体


% 设置最后一次更新时间
\usepackage{lastpage}
\usepackage{fancyhdr}
\pagestyle{fancy}
\fancyhf{}
\fancyfoot[L]{\footnotesize\textit{郭大为个人简历}}
\fancyfoot[R]{\footnotesize\textit{最后一次更新于: \today}}

% 调整页边距
\usepackage[scale=0.85]{geometry}
\setlength{\hintscolumnwidth}{3cm}						% if you want to change the width of the column with the dates
%\AtBeginDocument{\setlength{\maketitlenamewidth}{6cm}}  % only for the classic theme, if you want to change the width of your name placeholder (to leave more space for your address details
%\AtBeginDocument{\recomputelengths}                     % required when changes are made to page layout lengths

% 在其他左侧插入标志
\usepackage{manfnt}
\newcommand{\hello}{{\tiny\textdbend}}

\linespread{1.2}

% 设置超链接字体颜色
\AtBeginDocument{
    \hypersetup{colorlinks,urlcolor=blue}
}


% 设置个人信息
% \name{\textbf{郭大为}}{\textbf{}}
\name{郭大为}{}
% 下面所有信息都是可选的,如果不需要,可以删掉
\title{\textbf{湖北/男}}
\address{}{广州市海珠区} 
\mobile{188-1411-5785} 
%\phone{phone (optional)}
%\fax{fax (optional)} 
\email{guodw3@mail2.sysu.edu.cn} 
\homepage{david-guo.github.io}          
%\extrainfo{♂34岁}       
\photo[64pt]{qr}
% ‘64pt’是图片必须压缩至的高度、‘0.4pt‘是图片边框的宽度 (如不需要可调节至0pt)
% ’picture‘ 是图片文件的名字;可选项、如不需要可删除本行

\social[github]{David-Guo}
\quote{\textit{求职意向:C++工程师、后端开发}}

% to show numerical labels in the bibliography; only useful if you make citations in your resume
%\makeatletter
%\renewcommand*{\bibliographyitemlabel}{\@biblabel{\arabic{enumiv}}}
%\makeatother

% bibliography with mutiple entries
% \usepackage{multibib}
% \newcites{book,misc}{{Books},{Others}}

%\nopagenumbers{}                             % uncomment to suppress automatic page numbering for CVs longer than one page
%----------------------------------------------------------------------------------
%            content
%----------------------------------------------------------------------------------
\begin{document}
\maketitle
\vspace{-3em}      %缩小段落的间距

\section{\textbf{教育背景}}
\cventry{2013年 -- 2017年}{学士学位}{中山大学}{信息与计算科学专业}{专业 top 7.5\%}{}
\vspace{-1em}

\section{\textbf{社区}}
\cvline{CSDN博客}{\small{\url{blog.csdn.net/u012675539}}, 技术博客}
\cvline{GitHub}{\small {\url{github.com/David-Guo}}, 活跃}
\cvline{网页简历}{\small{\url{david-guo.github.io/myresume/resume.pdf}},简要介绍}
\vspace{-1em}      %缩小段落的间距

\section{\textbf{实习经历}}
\cventry{2016/05--2016/08}{腾讯微信}{基础产品部}{C++后台开发}{}
{接手多个模块的开发,具有有代表性的是:
\begin{itemize}
\item 布隆过滤器模块,解决海量实时UV计算问题,记录一个用户15天内是否命中某一广告。压测QPS可达 150w/s。请求分发到三台主机,支持机器扩容,缩容,纪录binlog, 支持对帐,读写容灾.
\item 广告投放记录kv存储模块,将用户id,广告id,投放时间等信息存储到 kv 数据库中,并能通过用户 id 检索出对该用户投放的所有记录.
\item 实验系统门户网站 CGI,为门户网站增添了部分页面的后台CGI.
\item 实验系统门户网站数据库模块,用C++封装Mysql语句提供接口给 CGI 调用.
\item 微信App发现页面附近功能开发,入口在朋友圈下一栏,根据用户地理位置系统推荐美食,主要负责删除评论、衔接配置系统等功能后台逻辑.
\end{itemize}
}

\section{\textbf{项目经历}}
\cventry{2016/08}
{Flask \& React 留言板应用}
{react,python}
{}{}
{响应式设计,前后端分离。前端使用 webpack 构建, react 处理表单,AJAX 传递数据,后端使用 flask 框架,shelve 模块储存数据。源码地址:\url{github.com/David-Guo/message-board}}

\cventry{2016/07}
{花草识别}
{python,caffe}
{科研项目}{}
{常见花识别的在线演示,计算部分使用 caffe 在 offord102 数据集上训练CNN经典模型 AlexNet,服务器端采用了基于 python 的 web开发框架 cherrpy}

\cventry{2016/03}
{二手交易论坛}
{python}
{个人项目}{}
{后端使用 flask 前端使用 bootstrap ,首页使用响应式设计。采用 Nginx + Supervisrod + Gunicorn + Mysql 部署在云平台。
支持多人注册登录,在线发表文章或评论,上传图片,关注他人,查看关注者;修改个人资料、密码、头像;首页以列表分页方式展示所有文章,并且仅显示文章中的 50 个字,出现图片的则以占位符图片替代。
\\源码地址:\url{github.com/David-Guo/flaskforum}
\\部署地址:\url{sysufm.tk}}

\cventry{2015/12}
{简单Http server\&CGI}
{c++}
{课程项目}{}
{用linux原生socket接口解析http协议并发接受Get请求,开启新的 sock 网络连接访问 ras(remote assess shell) 服务器,读取本地的指令批文件传送给 ras(remote assess shell) ,取回从 ras 计算得到的数据打包成 html 页面返回给浏览器。有 linux 版和 windows 版。
\\源码地址1:\url{github.com/David-Guo/winsockHttpServer}
\\源码地址2:\url{github.com/David-Guo/cgi_http}}

\cventry{2015/10}
{shell 解释器}
{C++}
{课程项目}{}
{实现一个简单的shell解释器,支持如下功能:
  	解释命令行输入执行指令、
    shell内建命令cd exit、
    pipe 管道、
    fg 前台执行命令、
    bg 后台执行命令、
    彩色提示符、
    Ctrl -Z Ctrl -C 信号处理。
\\源码地址:\url{github.com/David-Guo/myshell}}

\cventry{2014/12}
{基于PCA的人脸识别GUI实现}
{Matlab}
{课程项目}{}
{《机器学习》课程上,在查看相关文献,理解PCA与最小距离分类器算法原理后,自己动手用熟悉的MATLAB简单实现了一个GUI界面程序。在Yale人脸库上实验,分类器效果比较理想,准确率高达90\%多。
\\项目源码:\url{github.com/David-Guo/face_recongnize}}

\cventry{2014/09}
{基于心电信号的辅助诊断软件}
{MFC、OpenGL}
{科研项目}{}
{主要职责:
\begin{itemize}
	\item MFC界面设计实现
	\item DLL封装与数据库构建
	\item 实现多层次熵核心算法,嵌入SVM工具箱
\end{itemize}
核心算法地址:{\url{david-guo.github.io/2015/05/22/entropy_code/}}}

\section{竞赛经历}
\cventry{2015.5}
{世界大学生超级计算机竞赛}{}{}{决赛一等奖}
{主要负责:在队伍搭建起的Centos九计算节点集群上运行、采用大页面,ramdisk 方式优化wrf程序。}
\cventry{2014.9}
{2014全国大学生数学建模竞赛}{}{}{三等奖}{}
\cventry{2014.10}
{2014全国大学生数学竞赛}{}{}{二等奖}{}

\section{\textbf{IT技能}}
\cvline{编程语言}{ \small MATLAB == C >= C++ >= Python == JS == shell脚本 }
\cvline{web框架}{ \small Flask,Django,React }
\cvline{开发工具}{\small Linux, Gcc, Vim, Git, VS }
\cvline{文档编辑}{\small \LaTeX, Markdown}
\vspace{-1em}      %缩小段落的间距

\section{\textbf{获奖情况}}
\cvlistdoubleitem{校内二等奖学金(2014、2015)}{国家励志奖学金(2014)}
\cvlistdoubleitem{中国友好和平发展基金会奖学金(2015)}{}
\vspace{-1em}      %缩小段落的间距

\section{\textbf{外语}}
\cvline{英语}{\small CET-6,具备阅读专业英文文献、写作及翻译能力,英文博客翻译曾见诸于CSDN云计算首页}
%\cvlanguage{英语}{具备阅读专业英文文献能力及写作能力}{CET-6}
\vspace{-1em}      %缩小段落的间距

\section{\textbf{其他}}
\cvline{\hello}{靠谱,责任心强,乐于助人;熟悉 linux 编程API,熟悉TCP/IP协议族。数学功底扎实,擅长查看文献;轻度代码洁癖;具有较好的人际沟通、协调和组织能力;}
%\cvlistitem{}

\end{document}
